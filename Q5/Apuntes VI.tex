\documentclass{article}

\usepackage{color}
\usepackage{float}
\usepackage{xcolor}
\usepackage{import}
\usepackage{amssymb}
\usepackage{amsmath}
\usepackage{fancyhdr}
\usepackage{lastpage}
\usepackage{graphicx}
\usepackage{hyperref}
\usepackage{ragged2e}
\usepackage{blindtext}
\usepackage{multicol,caption}
\usepackage[toc,page]{appendix}
\usepackage[nodayofweek]{datetime}
\usepackage[a4paper, total={6in, 8in}]{geometry}

\hypersetup{
    colorlinks=true, %set true if you want colored links
    linktoc=all,     %set to all if you want both sections and subsections linked
    linkcolor=black,  %choose some color if you want links to stand out
}

\newenvironment{Figure}
  {\par\medskip\noindent\minipage{\linewidth}}
  {\endminipage\par\medskip}

\pagenumbering{gobble}
\pagestyle{fancy}
\setlength{\headheight}{13.07225pt}

\begin{document}

\begin{titlepage}
    \newcommand{\HRule}{\rule{\linewidth}{0.5mm}}
    
    \center % Centre everything on the page
    \textsc{\LARGE Universidad Politécnica de Cataluña}\\[1.5cm]
    \textsc{\Large Grado en Ciencia e Ingeniería de Datos}\\[0.5cm]
    \HRule\\[0.4cm]
    {\huge\bfseries Introducción a la Visualización}\\[0.4cm]
    \HRule\\[1.5cm]
    \vfill
    {\large\today}
\end{titlepage}

\newpage
\tableofcontents
\newpage

\fancyfoot{} % clears the settings for the footers
\fancyfoot[R]{Page \thepage \hspace{1pt} of \pageref{LastPage}}
\pagenumbering{arabic}

\section{Introduction}

\begin{itemize}
  \item \texttt{Expressiveness}: Show exactly the information of the data, nothing more and nothing else. It can be messured.
  \item \texttt{Effectiveness}: The visualization should be easy to understand in terms of the cognition capabilities of the human brain.  If the visualization reduces the required time to understand the data, not for a specific individual but for the whole group, then it is a good visualization.
  \item \texttt{Appropriateness}: The visualization should be appropriate for the data and the task. Cost-Value ratio should be considered.
\end{itemize}


% The process of gathering data can be a problem in itself for a visualization.

\newpage
\section{Homework}
\begin{itemize}
  \item How many variables (and which) are displayed in the following charts:
  
  \item The one shown at 2:26
  \textcolor{blue}{fertility rate (x-axis), life expectancy (y-axis), population (size of the bubble), continent (color) and time (animation)}

  \item The one shown at 13:07
  \textcolor{blue}{GDP per capita (x-axis), child survival (y-axis), population (size of the bubble), continent (color) and time (animation)}

  \item  Find an example of visual representation that does not effectively communicate the message (minute and second, and reason why)
  \textcolor{blue}{16:16. The use of the analogy to express who the public, internet and data can be confused and doenst really seem to be appropriate.}
  
  \item What happens in terms of variables when he “splits South Africa”?
  \textcolor{blue}{There is a difference in the quantiles of South Africa after spliting it. With a large separation from the poorest to the richest that cant be seen when the country is not split.}

  \item Bonus: Who is the “ghost”?
  \textcolor{blue}{The income distribution of China.}

\end{itemize}
\end{document}