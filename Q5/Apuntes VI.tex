\documentclass{article}

\usepackage{color}
\usepackage{float}
\usepackage{xcolor}
\usepackage{import}
\usepackage{amssymb}
\usepackage{amsmath}
\usepackage{fancyhdr}
\usepackage{lastpage}
\usepackage{graphicx}
\usepackage{hyperref}
\usepackage{ragged2e}
\usepackage{blindtext}
\usepackage{multicol,caption}
\usepackage[toc,page]{appendix}
\usepackage[nodayofweek]{datetime}
\usepackage[a4paper, total={6in, 8in}]{geometry}

\hypersetup{
    colorlinks=true, %set true if you want colored links
    linktoc=all,     %set to all if you want both sections and subsections linked
    linkcolor=black,  %choose some color if you want links to stand out
}

\newenvironment{Figure}
  {\par\medskip\noindent\minipage{\linewidth}}
  {\endminipage\par\medskip}

\pagenumbering{gobble}
\pagestyle{fancy}
\setlength{\headheight}{13.07225pt}

\begin{document}

\begin{titlepage}
    \newcommand{\HRule}{\rule{\linewidth}{0.5mm}}
    
    \center % Centre everything on the page
    \textsc{\LARGE Universidad Politécnica de Cataluña}\\[1.5cm]
    \textsc{\Large Grado en Ciencia e Ingeniería de Datos}\\[0.5cm]
    \HRule\\[0.4cm]
    {\huge\bfseries Introducción a la Visualización}\\[0.4cm]
    \HRule\\[1.5cm]
    \vfill
    {\large\today}
\end{titlepage}

\newpage
\tableofcontents
\newpage

\fancyfoot{} % clears the settings for the footers
\fancyfoot[R]{Page \thepage \hspace{1pt} of \pageref{LastPage}}
\pagenumbering{arabic}

\section{Introduction}

\begin{itemize}
  \item \texttt{Expressiveness}: Show exactly the information of the data, nothing more and nothing else. It can be messured.
  \item \texttt{Effectiveness}: The visualization should be easy to understand in terms of the cognition capabilities of the human brain.  If the visualization reduces the required time to understand the data, not for a specific individual but for the whole group, then it is a good visualization.
  \item \texttt{Appropriateness}: The visualization should be appropriate for the data and the task. Cost-Value ratio should be considered, mainly in time and space terms.
\end{itemize}

\noindent
Augment the capabilities of the human rather than replacing it by computation decision making. Visualization is related to understanding the underlying data by helping the user to understand data using their excellent perception capabilities. It helps the user to carry out tasks more effectively.

\subsection{Main applications}
The main applications of visualizations are: 
\begin{itemize}
    \item \textit{Explanatory:} present the results. Visualization is used for \textbf{presentation}. To communicate data and ideas, explain and inform providing evidence and influence and persuade. \textbf{Commonly only showing a few variables of the data.}
    \item \textit{Analysis:} Analyse hypothesis. The typical objectives are \textbf{showing many variables}, illustrate overview and detail to \textbf{facilitate comparison}. Presentation might choose some parts, \textbf{analysis will focus on all of them}.
    \item \textit{Exploratory:} Inspect the data to learn new things, get \textbf{insights.}
\end{itemize}

\subsection{Data Cleaning}
Dirty data in visualization may lead to wrong conclusions. It is important to clean the data before visualizing it. In many scenarios we have not control of the gathering of the data, which may cause a misleading visualization.
\begin{itemize}
  \item Magnitudes of the data might not be in the same scale or coincide due to the technologies used in the data collection.
  \item Suspicios data: Drops in the tendecy of accumulative data.
\end{itemize}

% The process of gathering data can be a problem in itself for a visualization.




\end{document}